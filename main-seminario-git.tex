% introduzione al corso di codifica del testo anno accademico 2018/2019

\documentclass{beamer}
    
%    \usepackage[english]{babel}
    %\usepackage[latin1]{inputenc}
    %\usepackage[T1]{fontenc}

\mode<presentation>{
  \setbeamertemplate{background canvas}[vertical shading]
  \usetheme{Berkeley}
  \useoutertheme{himinfolines}
}
  
\usepackage{ucs}
\usepackage[utf8]{inputenc}
\usepackage[english,polutonikogreek,italian,UKenglish,british]{babel}
\usepackage{graphicx}
\usepackage{colortbl}
\usepackage{multicol}
\usepackage{ulem}
\usepackage{verbatim}
\usepackage{alltt}
\usepackage{ccicons}
\usepackage{MnSymbol,wasysym}
\usepackage{tikzsymbols}
\usepackage{textcomp}
\usepackage{xmpincl}

\usepackage{parskip}
\setcounter{nframes}{70}
\setcounter{nframe}{1}
\setbeamercovered{dynamic}
\newenvironment{grcenv}{\begin{otherlanguage}{greek}}{\end{otherlanguage}}
\newcommand{\g}[1]{\textgreek{#1}}
\definecolor{darkgreen}{rgb}{0,0.5,0}
\definecolor{darkblue}{rgb}{0,0,0.5}
\definecolor{grey}{rgb}{0.5,0.5,0.5}
\setcounter{tocdepth}{5}

\makeatletter

\makeatother
%\includexmp{LicencesAndLicensing}

%frame00 metadata
	\title{Introduzione sistema git per edizioni collaborative}
	\author[A.M. Del Grosso]{Angelo Mario Del Grosso}
	\institute{DIGITAL TOOLS FOR HUMANISTS SUMMER SCHOOL 2019}
    %\institute{\texttt{angelo.delgrosso@ilc.cnr.it} \\\bigskip\textit{CNR-ILC-LicoLab} \\\bigskip\url{http://licolab.ilc.cnr.it/}}
    \institute{\texttt{angelo.delgrosso@ilc.cnr.it} \\\bigskip\textit{CNR-ILC-LicoLab}}
    \date{Istituto di Linguistica Computazionale ``A. Zampolli'', \today}
    \AtBeginSection[]{
    \begin{frame}<beamer>
    \addtocounter{nframe}{1}
    \footnotesize
    \frametitle{Progress status}
    \tableofcontents[currentsection,hideothersubsections]
    \end{frame}
    }

\begin{document}

\begin{frame}
	\maketitle
\end{frame}

\begin{frame}
	\frametitle{Outline}
	\tableofcontents
\end{frame}

\section{Self Introduction}

\begin{frame}
	\frametitle{What is my work about}
	\addtocounter{nframe}{1}

	\begin{block}{Digital and Computational Philology}
		%slide di presentazione: chi sono, piccola bio, di cosa mi occupo
		%\\prendere dal curriculum alcuni pezzi mettere mail istituzionale ed eventualmente telefono
		Analisi, progettazione e sviluppo di componenti software per sistemi di linguistica e filologia digitale/computazionale volti alla produzione, rappresentazione, trattamento, fruizione e interrogazione di testi di tradizione medievale, a stampa e di autori moderni e contemporanei.
	\end{block}

	% \begin{block}{Modelli Object Oriented per il Textual Scholarship}
	% 	Impiego delle nuove tecnologie nell’ambito delle Digital Humanities (DH) per la progettazione object–oriented di strumenti digitali Web–based rispondenti alle esigenze degli utenti accademici, studenti e sviluppatori.
	% \end{block}


\end{frame}

\begin{frame}
	\frametitle{Topic of the talk}
	\addtocounter{nframe}{1}

	\begin{center}
		\includegraphics[width=.7\textwidth]{./imgs/git-github.jpeg}
	\end{center}

	%\begin{itemize}
	%	\item<1-> Introduzione
	%	\item<2-> Codifica dei Caratteri
	%   \item<3-> Codifica dei Testi
	%   \item<4-> Ecosistema XML (Linee Guida TEI)
	%   \item<5-> Conclusioni
	%\end{itemize}

\end{frame}


\begin{frame}
	\frametitle{First of all}
	\addtocounter{nframe}{1}

	\begin{itemize}
		\item<1-> Version Control Systems (VCSs) 
		\item<2-> Git usage through the main CLI commands 
		\item<3-> Cloning, modifying, contributing, diffing, logging
		\item<4-> Working with remotes
		\item<5-> GitHub hosting service (little tips on projects and organization)
		\item<6-> Branching model (\emph{NO WITHIN THIS TALK})
		\item<7-> Adavanced git tools (\emph{NO WITHIN THIS TALK})
	\end{itemize}

\end{frame}

https://git-scm.com/book/en/v2

\begin{frame}
	\frametitle{Topic of the talk}
	\addtocounter{nframe}{1}

	\begin{block}{Book which this workshop is derived from}
		\begin{center}
			\includegraphics[width=.7\textwidth]{./imgs/GIT-BOOK.png}
		\end{center}
	\end{block}

\end{frame}

\begin{frame}
	\frametitle{Topic of the workshop}
	\addtocounter{nframe}{1}

	\begin{block}{Working Session - Example of using Git}
		\begin{center}
			\includegraphics[width=.7\textwidth]{./imgs/Diffing-line-word.png}
		\end{center}
	\end{block}

\end{frame}

\section{VCS Introduction}
% sezione intro frame 
\begin{frame}
    \frametitle{Git and GitHub}
    \framesubtitle{Getting started with Git}
    \addtocounter{nframe}{1}
    
    \begin{block}{--}
    \end{block}

    \begin{block}{--}
    \end{block}

\end{frame}

% sezione intro frame 00
\begin{frame}
    \frametitle{Git intro}
    \framesubtitle{Getting started with Git}
    \addtocounter{nframe}{1}
    
    \begin{block}{VCS}
        Version control (VCS) is a system that records changes to a file or set of files over time so that you can recall specific versions later.
    \end{block}

    \begin{block}{Benefits}
        \begin{itemize}
            \item It allows you to revert selected files back to a previous state
            \item compare changes over time
            \item who last modified something that might be causing a problem
            \item \dots
        \end{itemize}
    \end{block}

\end{frame}


% sezione intro frame 01
\begin{frame}
    \frametitle{Git and GitHub}
    \framesubtitle{Getting started with Git}
    \addtocounter{nframe}{1}
    
    \begin{block}{--}
        Using a VCS also generally means that if you screw things up or lose files, you can easily recover
    \end{block}

    \begin{block}{--}
    \end{block}

\end{frame}

% sezione intro frame 01
\begin{frame}
    \frametitle{Git and GitHub}
    \framesubtitle{Getting started with Git}
    \addtocounter{nframe}{1}
    
    \begin{block}{Different VCS Architectures}
        \begin{itemize}
            \item Local Version Control System (RCS)
            \item Centralized Version Control System (CVS, SVN)
            \item Distributed Version Control System (GIT, Mercurial)
        \end{itemize}
    \end{block}

\end{frame}

% sezione intro frame 01
\begin{frame}
    \frametitle{Git and GitHub}
    \framesubtitle{Getting started with Git}
    \addtocounter{nframe}{1}
    
    \begin{block}{Local Version Control System}
        \begin{center}

            \includegraphics[width=.7\textwidth]{imgs/local.png}
    
        \end{center}
    
    \end{block}

    \textit{Simple database that kept all the changes to files under revision control}

\end{frame}

% sezione intro frame 01
\begin{frame}
    \frametitle{Git and GitHub}
    \framesubtitle{Getting started with Git}
    \addtocounter{nframe}{1}
    
    \begin{block}{Centralized Version Control System}
        \begin{center}

            \includegraphics[width=.7\textwidth]{imgs/centralized.png}
    
        \end{center}
    
    \end{block}

    \textit{Need to collaborate: single server that contains all the versioned files}

\end{frame}


% sezione intro frame 01
\begin{frame}
    \frametitle{Git and GitHub}
    \framesubtitle{Getting started with Git}
    \addtocounter{nframe}{1}
    
    \begin{block}{Distributed Version Control System}
        \begin{center}

            \includegraphics[width=.7\textwidth]{imgs/distributed.png}
    
        \end{center}
    
    \end{block}

    \textit{Client repositories can be copied back up to the server to restore it}

\end{frame}

\begin{frame}
    \frametitle{Git and GitHub}
    \framesubtitle{Getting started with Git}
    \addtocounter{nframe}{1}
    
    \begin{block}{GIT DVCS}
       \begin{itemize}
           \item Started by Linux community
           \item Fast and efficient 
           \item Simple design
           \item non-linear development
           \item fully distributed
           \item handle large projects
           \item easy to use
       \end{itemize}
    
    \end{block}

\end{frame}

\begin{frame}
    \frametitle{Git and GitHub}
    \framesubtitle{Getting started with Git}
    \addtocounter{nframe}{1}
    
    \begin{block}{GIT DVCS}
        With Git, every time you commit, or save the state of your project, Git basically \textbf{takes a picture of what all your files look like} at that moment and stores a \textbf{reference to that snapshot}.    
    \end{block}

\end{frame}

\begin{frame}
    \frametitle{Git and GitHub}
    \framesubtitle{Getting started with Git}
    \addtocounter{nframe}{1}
    
    \begin{block}{GIT DVCS}
        Everything in git is \textbf{checksummed before it is stored} and is then referred to by that checksum
    \end{block}

    \begin{block}{GIT DVCS}
        40-character string composed of hexadecimal characters
    \end{block}
   
    \textit{a62bc012b405ee47d26b695708063a9f2ffad243}

\end{frame}




\begin{frame}
    \frametitle{Git and GitHub}
    \framesubtitle{Getting started with Git}
    \addtocounter{nframe}{1}
    
    \begin{block}{GIT DVCS}
        \begin{center}

            \includegraphics[width=.7\textwidth]{imgs/snapshots-git.png}
    
        \end{center}
    
    \end{block}
    

\end{frame}

\begin{frame}
    \frametitle{Git and GitHub}
    \framesubtitle{Getting started with Git}
    \addtocounter{nframe}{1}
    
    \textbf{Git has three main states that your files can reside in}

    \begin{block}{GIT DVCS}
       \begin{itemize}
           \item committed
           \item modified 
           \item staged
       \end{itemize}
    
    \end{block}

\end{frame}

\begin{frame}
    \frametitle{Git and GitHub}
    \framesubtitle{Getting started with Git}
    \addtocounter{nframe}{1}
    
    \begin{block}{GIT Areas}
        \begin{center}

            \includegraphics[width=.7\textwidth]{imgs/git-areas.png}
    
        \end{center}
    
    \end{block}
    

\end{frame}

\begin{frame}
    \frametitle{Git and GitHub}
    \framesubtitle{Getting started with Git}
    \addtocounter{nframe}{1}
    
    \textbf{Git has three main states that your files can reside in}

    \begin{block}{GIT local workflow}
       \begin{itemize}
           \item modify files in your working tree
           \item stage just those changes you want to be part of your next commit 
           \item do a commit which stores that snapshot permanently to your git directory
       \end{itemize}
    
    \end{block}

\end{frame}


\section{git environment by command line interface}
%\begin{frame}
    \frametitle{Git and GitHub}
    \framesubtitle{Getting started with Git}
    \addtocounter{nframe}{1}
    
	\begin{block}{Git command line environment}
		The command line is the only place you can run all Git commands.
    \end{block}

	\begin{block}{GUIs environment}
		GUIs implement only a partial subset of Git functionality for simplicity
    \end{block}
	

\end{frame}

\begin{frame}
	\frametitle{Git and GitHub}
    \framesubtitle{Getting started with Git}
    \addtocounter{nframe}{1}

	\begin{center}
		\includegraphics[width=.8\textwidth]{imgs/git-commands.png}
	\end{center}

\end{frame}

\begin{frame}
    \frametitle{Git and GitHub}
    \framesubtitle{Getting started with Git}
    \addtocounter{nframe}{1}
	
	\emph{Git comes with a tool called \textbf{git config} that lets you get and set configuration variables that control all aspects of how Git looks and operates}

	\begin{block}{git config}
		\begin{itemize}
			\item system (all users, all repositories)
			\item global (all repositories, single user)
			\item local (single repository, single user)
		\end{itemize}
    \end{block}

\end{frame}

\begin{frame}
    \frametitle{Git and GitHub}
    \framesubtitle{Getting started with Git}
    \addtocounter{nframe}{1}
	
	\emph{The first thing you should do when you install Git is to set your \textbf{user} name and \textbf{email} address
	}

	\begin{block}{git config}
		\begin{itemize}
			\item \texttt{git config --global user.name "Angelo Mario Del Grosso"}
			\item \texttt{git config --global user.email "angelo.delgrosso@ilc.cnr.it"}
		\end{itemize}
    \end{block}

\end{frame}

\begin{frame}
	\frametitle{Git and GitHub}
    \framesubtitle{Checking Your Settings}
    \addtocounter{nframe}{1}

	\begin{center}
		\includegraphics[width=.8\textwidth]{imgs/git-config.png}
	\end{center}

\end{frame}

\begin{frame}
    \frametitle{Git and GitHub}
    \framesubtitle{Getting started with Git}
    \addtocounter{nframe}{1}
	
	\begin{block}{help while using git}
		\begin{itemize}
			\item \texttt{git help <verb>}
			\item \texttt{man git-<verb>}
			\item \texttt{git <verb> --help}
			\item \texttt{git <verb> -h}
		\end{itemize}
    \end{block}

\end{frame}


\begin{frame}
    \frametitle{Git and GitHub}
    \framesubtitle{Getting started with Git}
    \addtocounter{nframe}{1}
	
	\begin{block}{fundamental capabilities}
		\begin{itemize}
			\item configure and initialize a repository
			\item tracking files
			\item stage and commit changes
			\item ignore certain files and file patterns
			\item undo mistakes
			\item browse the history
			\item view changes
			\item push and pull from remote repositories
		\end{itemize}
    \end{block}

\end{frame}

\begin{frame}
    \frametitle{Git and GitHub}
    \framesubtitle{Getting started with Git}
    \addtocounter{nframe}{1}
	
	\begin{block}{git repository}
		\begin{itemize}
			\item local directory that is not under version control, and turn it into a git repository
			\item clone an existing Git repository from elsewhere
		\end{itemize}
    \end{block}

\end{frame}

\begin{frame}
    \frametitle{Git and GitHub}
    \framesubtitle{Getting started with Git}
    \addtocounter{nframe}{1}
	
	\begin{block}{git repository init}
		\begin{itemize}
			\item \texttt{git init}
			\item \texttt{git clone <URL> <DIR>}
		\end{itemize}
    \end{block}

\end{frame}


\begin{frame}
    \frametitle{Git and GitHub}
    \framesubtitle{Getting started with Git}
    \addtocounter{nframe}{1}
	
	\begin{block}{git repository init}
		After \textbf{init} nothing in the project is tracked yet.\\
		Need to begin tracking those files and do an initial commit.
	\end{block}

	\begin{block}{specify the files you want to track}
		\begin{itemize}
			\item git add <FILE(S)>
			\item git commit -m "<MESSAGE>"
		\end{itemize}
	\end{block}
	
	\textit{Git repository with tracked files and an initial commit.}
    
\end{frame}


\begin{frame}
    \frametitle{Git and GitHub}
    \framesubtitle{Getting started with Git}
    \addtocounter{nframe}{1}
	
	\begin{block}{git clone repository}
		Every version of every file for the history of the project is pulled down by default when you run \texttt{git clone}
    \end{block}

\end{frame}

\begin{frame}
    \frametitle{Git and GitHub}
    \framesubtitle{Getting started with Git}
    \addtocounter{nframe}{1}
	
	\begin{block}{git repository init}
		Each file in your working directory can be in one of two states
	\end{block}

	\begin{block}{track files}
		\begin{itemize}
			\item tracked
			\item untracked
		\end{itemize}
	\end{block}
	
	\textit{Tracked files are files that were in the last snapshot; they can be \textbf{unmodified}, \textbf{modified}, or \textbf{staged}}
    
\end{frame}

\begin{frame}
	\frametitle{Git and GitHub}
    \framesubtitle{track files}
    \addtocounter{nframe}{1}

	\begin{center}
		\includegraphics[width=.8\textwidth]{imgs/git-lifecycle-files.png}
	\end{center}

\end{frame}

\begin{frame}
	\frametitle{Git and GitHub}
    \framesubtitle{status files}
    \addtocounter{nframe}{1}

	\textit{To determine which files are in which state: \textbf{the git status command}}

	\begin{center}
		\includegraphics[width=.8\textwidth]{imgs/git-status.png}
	\end{center}

\end{frame}

\begin{frame}
	\frametitle{Git and GitHub}
    \framesubtitle{adding files}
    \addtocounter{nframe}{1}


	\begin{block}{git add}
		In order to begin tracking a new file, you use the \textbf{command git add}
	\end{block}

	\begin{block}{git add}
		file is now \textbf{tracked} and \textbf{staged} to be \textbf{committed}
	\end{block}

	\textit{The git add command \textbf{takes a path name} for either a file or a directory}
	

\end{frame}

\begin{frame}
	\frametitle{Git and GitHub}
    \framesubtitle{adding files}
    \addtocounter{nframe}{1}


	\begin{block}{git add}
		File that is tracked has been modified in the working directory but not yet staged
	\end{block}

	\begin{block}{git add}
		To stage a modified tracked file, you have to run the \textbf{git add command} again.
	\end{block}
	
	\textit{After git add, the files are staged and will go into your next commit}	

\end{frame}

\begin{frame}
	\frametitle{Git and GitHub}
    \framesubtitle{adding files}
    \addtocounter{nframe}{1}


	\begin{block}{git add}
		If you modify a file after you run git add, you have to run git add again to stage the latest version of the file
	\end{block}

	\begin{center}
		\includegraphics[width=.8\textwidth]{imgs/git-add-modify.png}
	\end{center}

\end{frame}

\begin{frame}
	\frametitle{Git and GitHub}
    \framesubtitle{ignoring files}
    \addtocounter{nframe}{1}


	\begin{block}{.gitignore file}
		If you’ll have a class of files that you don’t want to track
	\end{block}

	\begin{block}{.gitignore file}
		you can create a file listing patterns to match them named \textbf{.gitignore}.
	\end{block}

\end{frame}

\begin{frame}
	\frametitle{Git and GitHub}
    \framesubtitle{ignoring files}
    \addtocounter{nframe}{1}

	\begin{center}
		\texttt{more .gitignore}
	\end{center}
	
	\begin{center}
		\includegraphics[width=.8\textwidth]{imgs/gitignore.png}
	\end{center}

\end{frame}

\begin{frame}
	\frametitle{Git and GitHub}
    \framesubtitle{viewing files}
    \addtocounter{nframe}{1}

	\begin{block}{git diff}
		know exactly what you changed, not just which files were changed\\
		by using the \textbf{git diff command}
	\end{block}

	\begin{block}{git diff}
		\begin{itemize}
			\item What have you changed but not yet staged (\texttt{git diff})
			\item what have you staged that you are about to commit (\texttt{git diff --staged})
		\end{itemize}
	
	\end{block}

\end{frame}

\begin{frame}
	\frametitle{Git and GitHub}
    \framesubtitle{viewing files}
    \addtocounter{nframe}{1}
	
	\begin{center}
		\includegraphics[width=.8\textwidth]{imgs/Diffing-line-word.png}
	\end{center}

\end{frame}

\begin{frame}
	\frametitle{Git and GitHub}
    \framesubtitle{committing files}
    \addtocounter{nframe}{1}

	\begin{block}{git commit}
		Any files you have created or modified that you haven’t run git add on since you edited them — won’t go into the commit.
	\end{block}

	\begin{block}{git commit}
		\begin{itemize}
			\item the simplest way to commit is to type (\texttt{git commit})
			\item type your commit message inline  (\texttt{git commit -m "message"})
		\end{itemize}
	
	\end{block}

\end{frame}

\begin{frame}
	\frametitle{Git and GitHub}
    \framesubtitle{committing files}
    \addtocounter{nframe}{1}

	\begin{block}{git commit}
		Every time you perform a commit, you’re recording a snapshot of your project that you can revert to or compare to later.
	\end{block}

	\begin{center}
		\includegraphics[width=.8\textwidth]{imgs/git-commit.png}
	\end{center}

\end{frame}

\begin{frame}
	\frametitle{Git and GitHub}
    \framesubtitle{removing files}
    \addtocounter{nframe}{1}

	\begin{block}{git rm}
		To remove a file from git, you have to remove it from your tracked files
	\end{block}

	\begin{block}{git rm}
		\begin{itemize}
			\item \texttt{git rm <FILE>}
			\item \texttt{git rm -f <FILE>}
			\item \texttt{git rm --cached <FILE>}
		\end{itemize}
	
	\end{block}

\end{frame}

\begin{frame}
	\frametitle{Git and GitHub}
    \framesubtitle{moving files}
    \addtocounter{nframe}{1}

	\begin{block}{git mv}
		If you rename a file in Git, no metadata is stored in Git that tells it you renamed the file
	\end{block}

	\begin{block}{git mv}
		\begin{itemize}
			\item \texttt{git mv <FILE_FROM> <FILE_TO>}
		\end{itemize}
	\end{block}

	\begin{block}{git mv}
		\begin{itemize}
			\item \texttt{mv <FILE_FROM> <FILE_TO>}
			\item \texttt{git rm <FILE_FROM>}
			\item \texttt{git add <FILE_TO>}
		\end{itemize}
	\end{block}
\end{frame}

\begin{frame}
	\frametitle{Git and GitHub}
    \framesubtitle{History of commits}
    \addtocounter{nframe}{1}

	\begin{block}{git log}
		\textbf{git log} lists the commits made in that repository in reverse chronological order, each commit with its checksum hash string, author’s name and email, date, the commit message.
	\end{block}

	\begin{block}{git log}
		\begin{itemize}
			\item \texttt{git log <options>}
		\end{itemize}
	\end{block}

\end{frame}

\begin{frame}
	\frametitle{Git and GitHub}
    \framesubtitle{History of commits}
    \addtocounter{nframe}{1}

	\begin{block}{git log}
		if you want to see some abbreviated stats for each commit, you can use \textbf{the --stat option}
	\end{block}

	\begin{block}{git log}
		\begin{itemize}
			\item \texttt{git log --stat}
		\end{itemize}
	\end{block}

\end{frame}

\begin{frame}
	\frametitle{Git and GitHub}
    \framesubtitle{History of commits}
    \addtocounter{nframe}{1}

	\begin{block}{git log --stat}
		\begin{center}
			\includegraphics[width=.9\textwidth]{imgs/git-log-stats.png}
		\end{center}
	\end{block}

\end{frame}

\begin{frame}
	\frametitle{Git and GitHub}
    \framesubtitle{History of commits}
    \addtocounter{nframe}{1}

	\begin{block}{git log options}
		\begin{center}
			%\includegraphics[width=.9\textwidth]{imgs/git-log-stats.png}
			\includegraphics[width=.9\textwidth]{imgs/git-log-options.png}
		\end{center}
	\end{block}

\end{frame}

\begin{frame}
	\frametitle{Git and GitHub}
    \framesubtitle{History of commits}
    \addtocounter{nframe}{1}

	\begin{block}{git log --pretty}
		\begin{center}
			%\includegraphics[width=.9\textwidth]{imgs/git-log-stats.png}
			\includegraphics[width=.9\textwidth]{imgs/git-log-pretty.png}
		\end{center}
	\end{block}

\end{frame}

\begin{frame}
	\frametitle{Git and GitHub}
    \framesubtitle{History of commits}
    \addtocounter{nframe}{1}

	\begin{block}{git log limit options}
		\begin{center}
			\includegraphics[width=.9\textwidth]{imgs/git-log-stats.png}
			%\includegraphics[width=.9\textwidth]{imgs/git-log-limits.png}
		\end{center}
	\end{block}

\end{frame}

\begin{frame}
	\frametitle{Git and GitHub}
    \framesubtitle{History of commits}
    \addtocounter{nframe}{1}

	\texttt{git log --pretty=``%h: %an -- %s'' --no-merges}
	\begin{block}{git log --pretty}
		\begin{center}
			\includegraphics[width=.9\textwidth]{imgs/git-log-out.png}
		\end{center}
	\end{block}

\end{frame}

\begin{frame}
	\frametitle{Git and GitHub}
    \framesubtitle{Undoing things}
    \addtocounter{nframe}{1}

	\begin{block}{amend option}
		If you commit too early and possibly forget to add some files, make the additional changes you forgot, stage them, and \textbf{commit again using the --amend option}.
		\\ You end up with a single commit — the \textit{second commit replaces the first one}.
	\end{block}

	\begin{block}{git log}
		\begin{itemize}
			\item \texttt{git commit --amend [-m "MESSAGE"]}
		\end{itemize}
	\end{block}

\end{frame}

\begin{frame}
	\frametitle{Git and GitHub}
    \framesubtitle{Undoing things}
    \addtocounter{nframe}{1}

	\begin{block}{unstage and discard changes}
		How can you unstage a file or revert it back to what it looked like when you last committed.
	\end{block}

	\begin{block}{git reset and checkout}
		\begin{itemize}
			\item \texttt{git reset HEAD <FILE>} (unstage file)
			\item \texttt{git checkout -- <FILE>} (discard changes)
		\end{itemize}
	\end{block}

\end{frame}

\begin{frame}
	\frametitle{Git and GitHub}
    \framesubtitle{Working with Remotes}
    \addtocounter{nframe}{1}

	\begin{block}{Remote Repositories}
		Remote repositories are versions of your project that are hosted on the Internet
	\end{block}

	\begin{block}{Remote Repositories}
		Collaborating with others involves managing remote repositories. \\
		This entails \textbf{pushing} and \textbf{pulling} data to and from remote repositories when you need to share data.
	\end{block}

\end{frame}

\begin{frame}
	\frametitle{Git and GitHub}
    \framesubtitle{Working with Remotes}
    \addtocounter{nframe}{1}

	\begin{block}{capabilities}
		\begin{itemize}
			\item add remote repositories
			\item remove remotes
			\item manage various remote branches
			\item define them as being tracked or not
			\item pushing, pulling and fetching operations
		\end{itemize}
	\end{block}

\end{frame}

\begin{frame}
	\frametitle{Git and GitHub}
    \framesubtitle{Working with Remotes}
    \addtocounter{nframe}{1}

	\begin{block}{remote repositories}
		To see which remote servers you have configured, you can run the \textbf{git remote command}
	\end{block}

	\begin{block}{git remote}
		\begin{itemize}
			\item \texttt{git remote}
			\item \texttt{git remote -v}
		\end{itemize}
	\end{block}

\end{frame}

\begin{frame}
	\frametitle{Git and GitHub}
    \framesubtitle{Working with Remotes}
    \addtocounter{nframe}{1}

	\begin{block}{remote repositories}
		To add a new remote Git repository as a shortname you can reference easily, run \textbf{git remote add <shortname> <url>}:
	\end{block}

	\begin{block}{git remote}
		\begin{itemize}
			\item \texttt{git remote add upstream-edition https://github.com/angelodel80/myEditon}
		\end{itemize}
	\end{block}

	\textit{If you clone a repository, the command automatically adds that remote repository under the name “origin”}

\end{frame}

\begin{frame}
	\frametitle{Git and GitHub}
    \framesubtitle{Working with Remotes}
    \addtocounter{nframe}{1}

	\begin{block}{remote repositories}
		to get data from your remote projects, you can run the \textbf{git fetch command}. \\
		It’s important to note that the git fetch command only downloads the data to your local repository — it doesn't automatically merge it with any of your work or modify what you’re currently working on.
	\end{block}

	\begin{block}{git remote}
		\begin{itemize}
			\item \texttt{git fetch <remote>}
		\end{itemize}
	\end{block}

\end{frame}

\begin{frame}
	\frametitle{Git and GitHub}
    \framesubtitle{Working with Remotes}
    \addtocounter{nframe}{1}

	\begin{block}{remote repositories}
		When you have your project at a point that you want to share, you have to \textbf{push it upstream}. This pushes any commits you’ve done back up to the server if you have write access and if nobody has pushed in the meantime.
	\end{block}

	\begin{block}{git remote}
		\begin{itemize}
			\item \texttt{git push <remote> <branch>}
		\end{itemize}
	\end{block}

\end{frame}

\begin{frame}
	\frametitle{Git and GitHub}
    \framesubtitle{Working with Remotes}
    \addtocounter{nframe}{1}


	\begin{block}{git remote}
		\begin{itemize}
			\item \texttt{git remote show <remote>}
		\end{itemize}
	\end{block}


	\begin{block}{remote repositories}
		\begin{center}
			\includegraphics[width=.9\textwidth]{imgs/git-remote-show}
		\end{center}
	\end{block}

\end{frame}

\begin{frame}
	\frametitle{Git and GitHub}
    \framesubtitle{Working with Remotes}
    \addtocounter{nframe}{1}

	\begin{block}{remote repositories}
		You can run \textbf{git remote rename} to change a remote’s shortname, if you want to remove a remote repository you can either use \textbf{git remote remove} command or \textbf{git remote rm} command. 

	\end{block}

	\begin{block}{git remote}
		\begin{itemize}
			\item \texttt{git remote rename original upstream-edition}
			\item \texttt{git remote remove upstream-edition}
		\end{itemize}
	\end{block}

\end{frame}

\begin{frame}
	\frametitle{Git and GitHub}
    \framesubtitle{Tagging}
    \addtocounter{nframe}{1}

	\begin{block}{tag specific points}
		Git has the ability to \textbf{tag specific points} in a \textit{repository's history} as being important, e.g. mark release points. Git supports two types of tags: lightweight and annotated.
	\end{block}

	\begin{block}{git tag}
		\begin{itemize}
			\item \texttt{git tag [-l] [--list] <PATTERN>} (list tags)
			\item \texttt{git tag -a <TAG-NAME> -m "MESSAGGIO" } (create an annotated tag)
			\item \texttt{git show <TAG-NAME>} (show the tag data)
			\item \texttt{git push <REMOTE> <TAG-NAME>} (push tag)
			\item \texttt{git tag -d  <TAG-NAME>} (delate locally)
			\item \texttt{git push <REMOTE> --delete <TAG-NAME>} (delate remotelly)
		\end{itemize}
	\end{block}

\end{frame}

\begin{frame}
	\frametitle{Git and GitHub}
    \framesubtitle{Tagging}
    \addtocounter{nframe}{1}

	\begin{block}{tag specific points}
		If you want to view the versions of files a tag is pointing to, you can do a git checkout of that tag.\\
		This puts your repository in “detached HEAD” state, which has some ill side effects
	\end{block}

	\begin{block}{git tag}
		\begin{itemize}
			\item \texttt{git checkout <TAG-NAME>} (View the files in tag version)
		\end{itemize}
	\end{block}

\end{frame}

% frame 00
\begin{frame}
	\frametitle{Elementi di Codifica dei Caratteri}
	\framesubtitle{Definizioni}
	\addtocounter{nframe}{1}

	\begin{block}{Rappresentare il testo in formato digitale}
		L’adozione di metodologie informatiche per il trattamento dei testi richiede in primo luogo la disponibilità di un'adeguata rappresentazione dei dati testuali in formato digitale.
	\end{block}

\end{frame}

% frame 00b
\begin{frame}
	\frametitle{Elementi di Codifica dei Caratteri}
	\framesubtitle{Problemi di rappresentazione}
	\addtocounter{nframe}{1}

	\begin{center}
		\includegraphics[width=.9\textwidth]{imgs/SaussureTrascrizione.pdf}
	\end{center}

\end{frame}


% frame 01
\begin{frame}
	\frametitle{Elementi di Codifica dei Caratteri}
	\framesubtitle{Definizioni}
	\addtocounter{nframe}{1}

	\begin{block}{Perché è importante la codifica dei caratteri}
		La codifica dei caratteri costituisce il grado zero (basso livello) della rappresentazione di testi su supporto digitale.
		\begin{center}
			\textit{Le codifiche dei caratteri sono la base di qualsiasi schema di codifica testuale}.
		\end{center}
	\end{block}

	\begin{block}{Rappresentazione digitale dei caratteri}
		I caratteri vengono rappresentati all’interno di un elaboratore mediante una sequenza di codici binari formati da opportune disposizioni di cifre composte da 0 e 1: 01100001 \textit{lettera a}
	\end{block}

\end{frame}



% frame 03
\begin{frame}
	\frametitle{Elementi di Codifica dei Caratteri}
	\framesubtitle{American Standard Code for Information Interchange}
	\addtocounter{nframe}{1}

	\begin{block}{Tabella Code Page ASCII 7 bit}
		%immagine di esempio Code Page ASCII (cp1252)
		\begin{center}
			\includegraphics[width=.9\textwidth]{imgs/ascii-67.pdf}
		\end{center}

	\end{block}
	%\hline
	\begin{tiny}
		\begin{center}
			7 bit = 128 possibili caratteri; 32 caratteri di controllo; 96 caratteri effettivi
		\end{center}

	\end{tiny}

\end{frame}

% frame 0
\begin{frame}
	\frametitle{Elementi di Codifica dei Caratteri}
	\framesubtitle{Esempio codifica binaria}
	\addtocounter{nframe}{1}

	\begin{block}{codifica \textit{ciao mondo!} 7 bit ASCII}
		\begin{center}
			\textsc{6369 616f 206d 6f6e 646f 210a}
		\end{center}
	\end{block}

	\begin{block}{codifica \textit{ciao è mondo!} 8 bit ASCII}
		\begin{center}
			\textmd{6369 616f 20\textbf{e8} 206d 6f6e 646f 210a       }
		\end{center}
	\end{block}

	\begin{block}{codifica \textbf{ciao è mondo!} UNICODE UTF-8}
		\begin{center}
			6369 616f 20\textbf{c3 a8}20 6d6f 6e64 6f21 0a
		\end{center}
	\end{block}

\end{frame}

% frame 02
\begin{frame}
	\frametitle{Elementi di Codifica dei Caratteri}
	\framesubtitle{Definizioni}
	\addtocounter{nframe}{1}

	% \begin{block}{Character set, Code Set}
	%  - Character set
	%  - Code Set
	%  - Character encoding
	%  - Tabella del Code page
	% \end{block}

	\begin{description}
		\item [Character Set] Per le discipline che studiano i sistemi di scrittura e l'analisi del linguaggio naturale, un insieme di caratteri astratti è detto Character set (unità alfabetiche). Astratto perché non riguarda la rappresentazione materiale della forma sul supporto, ma è relativo alla forma mentale, fatta di simboli di codifica (referenti).
		\item [Coded Char Set] Per poter trattare un insieme di unità alfabetiche in formato digitale bisogna assegnare a ciascun carattere un numero intero non negativo detto code point.
		
	\end{description}

\end{frame}

% frame 02b
\begin{frame}
	\frametitle{Elementi di Codifica dei Caratteri}
	\framesubtitle{Definizioni}
	\addtocounter{nframe}{1}

	% \begin{block}{Character set, Code Set}
	%  - Character set
	%  - Code Set
	%  - Character encoding
	%  - Tabella del Code page
	% \end{block}

	\begin{description}
		\item [Character Encoding]  Il fine ultimo della codifica è quello di rappresentare una sequenza di caratteri in una sequenza di byte. La codifica di un carattere utilizza uno ``encoding schema'' che a sua volta mappa o trasforma ciascun code point in una sequenza di byte e quindi in ultima istanza in una sequenza di bit. 
		\item [Tabella del code page] Generalmente i code points sono espressi attraverso un sistema numerico esadecimale e disposti in una tabella di associazione.
	\end{description}

\end{frame}

% frame 02c
\begin{frame}
	\frametitle{Elementi di Codifica dei Caratteri}
	\framesubtitle{In sintesi}
	\addtocounter{nframe}{1}


	\begin{block}{Codifica dei caratteri}
		Quindi trasformare una sequenza di caratteri appartenenti ad un char set in una sequenza di byte (bit) significa prima di tutto trasformare/mappare ciascun carattere nel proprio corrispettivo code point e successivamente codificare/serializzare questo code point nella relativa sequenza di byte (bit).
	\end{block}

\end{frame}


% frame 0
\begin{frame}
	\frametitle{Elementi di Codifica dei Caratteri}
	\framesubtitle{Complessità e rappresentazione}
	\addtocounter{nframe}{1}

	\begin{block}{Complessità di rappresentazione universale dei caratteri}
		Se si considerano tutti i possibili alfabeti del mondo e le molteplici esigenze poste dalla scrittura delle fonti manoscritte antiche e medievali, ci si accorge che la realizzazione di un sistema universale per la codifica dei caratteri è un progetto molto complesso con svariate sfide da affrontare.
	\end{block}

\end{frame}

% frame 0
\begin{frame}
	\frametitle{Complessità della Codifica dei Caratteri}
	\framesubtitle{Un Esempio}
	\addtocounter{nframe}{1}

	\begin{center}
		\includegraphics[width=.9\textwidth]{imgs/SnippetRotulo.jpg}
	\end{center}

\end{frame}

% frame 0
\begin{frame}
	\frametitle{Complessità della Codifica dei Caratteri}
	\framesubtitle{Un Esempio}
	\addtocounter{nframe}{1}

	\begin{center}
		\includegraphics[width=.9\textwidth]{imgs/tavolettaArgilla.jpg}
	\end{center}

\end{frame}


\begin{frame}
	\frametitle{Elementi di Codifica dei Caratteri}
	\framesubtitle{Unicode}
	\addtocounter{nframe}{1}

	\begin{block}{Complessità di rappresentazione universale}
		Ad oggi, lo standard de facto per la codifica dei caratteri è lo UNICODE. Esso è in grado di codificare più di un milione di differenti unità alfabetiche, segni di interpunzione e diacritici, appartenenti a centinaia di diverse lingue.
	\end{block}

	\begin{block}{Complessità di rappresentazione universale}
		%(1.114.111)
		Unicode assegna i propri code point in un range che va da $0x0$ a $0x10FFFF$. In Unicode il code point viene  indicato con una ``U'' seguita da un segno ``+'' seguito a sua volta dall'esadecimale con padding del codice (es: U+0041 lettera A).
	\end{block}

\end{frame}

\begin{frame}
	\frametitle{Elementi di Codifica dei Caratteri}
	\framesubtitle{Unicode}
	\addtocounter{nframe}{1}

	\begin{block}{Unicode Transformation Format}
		Lo Unicode è un Coded Char Set e per essere concretamente serializzato su un supporto elettronico deve essere trasformato attraverso qualche tipo di schema di codifica.
		L'UTF (Unicode Transformation Format) mappa i code point Unicode in sequenze di byte (bit).
	\end{block}

	\begin{block}{UTF standards}
		Esistono tre tipi di schemi di codifica che vanno sotto il nome di UTF, ciascuno è identificato dal minimo numero di bit necessario a codificare ciascun code point: UTF-8; UTF-16; UTF-32. 
	\end{block}

\end{frame}



\section{gitHub host platform}
%\begin{frame}
	\frametitle{Github}
	\framesubtitle{Init a repository}
	\addtocounter{nframe}{1}
	
		\begin{block}{Github platform}
		GitHub is the largest host for git repositories. It is a central point of collaboration among developers.
 		\end{block}

		 \begin{block}{Github capabilities}
			Git hosting, issue tracking, code review, and other things
		\end{block}
	
\end{frame}
	
\begin{frame}
\frametitle{Github}
\framesubtitle{Init a repository}
\addtocounter{nframe}{1}

	\begin{center}
		\includegraphics[width=.9\textwidth]{imgs/GitHub-RepoInit.png}
	\end{center}

\end{frame}

\begin{frame}
	\frametitle{Github}
	\framesubtitle{adding collaborators}
	\addtocounter{nframe}{1}
	
	
		\begin{center}
			\includegraphics[width=.95\textwidth]{imgs/github-Collaborators.png}
		\end{center}
	
\end{frame}

\begin{frame}
		\frametitle{Github}
		\framesubtitle{Comments to content lines}
		\addtocounter{nframe}{1}
		
			\begin{center}
				\includegraphics[width=.95\textwidth]{imgs/github-Commento-codice.png}
			\end{center}

\end{frame}

\begin{frame}
	\frametitle{Github}
	\framesubtitle{Comments to content lines}
	\addtocounter{nframe}{1}
	
		\begin{center}
			\includegraphics[width=.95\textwidth]{imgs/github-CommentoNotifica.png}
		\end{center}

\end{frame}
	

% \begin{frame}
% 	\frametitle{Rappresentazione digitale dei testi}
% 	\framesubtitle{basso e alto livello di codifica}
% 	\addtocounter{nframe}{1}

% 	\begin{block}{Codificare un testo}
% 		La codifica dei caratteri evidentemente non esaurisce i problemi per una opportuna rappresentazione delle caratteristiche interne ed esterne di un testo.
%     \end{block}
    
%     \begin{block}{Codificare un testo}
% 		Difatti la codifica del testo è una questione molto più complessa di una semplice riproduzione meccanica di un dato.
% 	\end{block}


% \end{frame}


% \begin{frame}
% 	\frametitle{Rappresentazione digitale dei testi}
% 	\framesubtitle{basso e alto livello di codifica}
% 	\addtocounter{nframe}{1}

% 	\begin{block}{Rappresentare un testo}
		
% 			La rappresentazione digitale di un testo è una operazione intrinsecamente assai difficile perché coinvolge una pletora di aspetti, a varie dimensioni, a varie granularità e a vari livelli di astrazione sia teorici, sia metodologici, sia tecnologici e sia pratici.
		
% 	\end{block}

% \end{frame}

% \begin{frame}
% 	\frametitle{Rappresentazione digitale dei testi}
% 	\framesubtitle{basso e alto livello di codifica}
% 	\addtocounter{nframe}{1}

% 	\begin{block}{Rappresentare un testo}
% 		\textbf{
% 			Prima di poter fare qualsiasi ipotesi su come compiere una codifica di un testo e su come rappresentarlo digitalmente, bisogna stabilire cosa si intende per testo.
% 		}
% 	\end{block}

% \end{frame}


% \begin{frame}
% 	\frametitle{Rappresentazione digitale dei testi}
% 	\framesubtitle{Modello dati di un testo}
% 	\addtocounter{nframe}{1}

% 	\begin{block}{Un testo non ha una struttura rigida, predefinita: }
% 		\begin{itemize}

% 			\item Non è rappresentabile solo come un insieme di record di un archivio elettronico.
% 			\item Non è rappresentabile solo come un insieme di tabelle di una banca dati.
% 			\item Non è rappresentabile solo come un albero o un insieme di sotto-alberi
% 			\item Non è rappresentabile solo come un grafo o come un insieme di sotto grafi

% 		\end{itemize}

% 	\end{block}

% \end{frame}

% \begin{frame}
% 	\frametitle{Molteplici modelli per diverse esigenze}
% 	\framesubtitle{Strutture dato e testo}
% 	\addtocounter{nframe}{1}

% 	\begin{block}{La rappresentazione di un testo}
% 		\begin{itemize}
% 			 %mia slide sulle possibili rappresentazioni del testo
% 			\item modello lineare: sequenza di dati non strutturati
% 			\item modello  a record: enumerazione delle proprietà
% 			\item modello tabulare: insieme di dati omogenei
% 			\item modello ad albero: gerarchie di dati e insiemi di dati
% 			\item modello grafo: rete di strutture informative interconnesse tra loro
%         \end{itemize}
        
% 	\end{block}
% \end{frame}



% \begin{frame}
% 	\frametitle{Elementi di Codifica del testo}
% 	\framesubtitle{Formalismi}
% 	\addtocounter{nframe}{1}

% 	\begin{block}{Formati di rappresentazione}
% 		\begin{center}
% 			Un formato è un insieme di regole e convenzioni formali per rappresentare un insieme di dati, nel nostro caso un testo.
% 		\end{center}

% 	\end{block}

% 	\begin{block}{Importanza dei formati}
% 		\begin{center}
% 			Seppur isomorfi la scelta dei formati condiziona molto l'efficienza delle operazioni e l'efficacia delle dichiarazioni.
% 		\end{center}

% 	\end{block}


% \end{frame}

% %\begin{frame}
% %    \frametitle{Elementi di Codifica del testo}
% %    \framesubtitle{lista di formati}
% %    \addtocounter{nframe}{1}
% %   
% %    \begin{block}{Formati dato}
% %Data structures – CSV and tabular data
% %– JSON
% %– RDF
% %Plain text formats – Plain text
% %– TeX, LaTeX, etc.
% %– Markdown, CommonMark and wiki syntaxes
% %Markup formats
% %– HTML, HTML5
% %– XML
% %– HTML5+ Embedded annotations (e.g., HTML5 + RDFa)
% %– Markup spinoffs for overlapping (e.g. LMNL, TexMECS, etc.) 
% %
% %    \end{block}

% % \begin{block}{Riferimenti TEI}
% %     Capitolo sul character encoding e modulo Ganji 
% % \end{block}

% %\end{frame}

% \begin{frame}
% 	\frametitle{Elementi di Codifica del testo}
% 	\framesubtitle{Tabella Formalismi}
% 	\addtocounter{nframe}{1}

% 	\begin{block}{Formalismi}
% 		\begin{center}
% 			\includegraphics[width=.9\textwidth]{imgs/TabellaFormalismiCodificaTesto.png}
% 		\end{center}
% 	\end{block}
% 	courtesy of \textit{Fabio Vitali}

% \end{frame}


% \begin{frame}
% 	\frametitle{Elementi di Codifica del testo}
% 	\framesubtitle{Varietà di rappresentazione}
% 	\addtocounter{nframe}{1}

	
% 		\begin{center}
% 			\includegraphics[width=.9\textwidth]{imgs/dataModels-slide.png}
% 		\end{center}
	
	
% \end{frame}

% \begin{frame}
% 	\frametitle{Elementi di Codifica del testo}
% 	\framesubtitle{Esempio di codifica del testo utilizzando CSV}
% 	\addtocounter{nframe}{1}

		
% 			\includegraphics[width=1.1\textwidth]{imgs/VariRappresentazioniTesto.png}
		
	
% \end{frame}



% \begin{frame}
% 	\frametitle{Elementi di Codifica del testo}
% 	\framesubtitle{Formalismi}
% 	\addtocounter{nframe}{1}

% 	\begin{block}{Formati come formalismi}
% 		\begin{center}
% 			Data l'importanza metodologica il formato del dato diviene un vero e proprio formalismo, si parla cioè di linguaggi di codifica in quanto questi sistemi si basano su un insieme di istruzioni rigorose di codifica.
% 		\end{center}

% 	\end{block}

% \end{frame}



% \begin{frame}
% 	\frametitle{Elementi di Codifica del testo}
% 	\framesubtitle{Formalismi}
% 	\addtocounter{nframe}{1}

% 	\begin{block}{Formati e formalismi di codifica}

% 		Quindi ogni pezzo di informazione aggiunta ad un testo grezzo attraverso l'inserimento di dati metatestuali (markup, annotazione, codifica), constituisce il risultato di una analisi e di una interpretazione che è stata condotta (da un umano o da una macchina) al fine di esplicitare e rappresentare nel modo più accurato e completo possibile le informazioni da veicolare attraverso il formato digitale prescelto (anche in modo incrementale).


% 	\end{block}

% \end{frame}



 


 
% need to interact with GitHub at some point while using Git professionally

 
% This chapter is about using GitHub effectively.

 
% signing up for and managing an account

 
% creating and using Git repositories, common workflows to contribute to projects and to accept contributions to yours

 
% set up a free user account

 
% Sign up for GitHub

 
% GitHub will send you an email to verify the address you provided

 
% your dashboard page

 
% ready to use GitHub

 
% connect with Git repositories using the https:// protocol

 
% Next, if you wish, you can replace the avatar that is generated for you with an image of your choosing.

 
% people will see your avatar next to your username

 
% The way that GitHub maps your Git commits to your user is by email address

 
% extra security

 
% Two-factor Authentication

 
% Authentication is an authentication mechanism

 
% code in addition to your password whenever you log into GitHub.

 
% could be useful in helping you contribute to an existing project

 
% you don’t have push access, you can “fork” the project

 
% GitHub will make a copy of the project that is entirely yours

 
% it lives in your namespace, and you can push to it

 
% 170 about the change until the owner is happy with it

 
% In GitHub, a “fork” is simply the same project in your own namespace

 
% People can fork a project, push to it, and contribute their changes back to the original repository by creating what’s called a Pull Request

 
% This opens up a discussion thread with code review,

 
% the contributor can then communicate 170 about the change until the owner is happy with it

 
% at which point the owner can merge it in

 
% “Fork” button

 
% with your own writeable copy of the code.

 
% GitHub is designed around a particular collaboration workflow, centered on Pull Requests.

 
% teams use GitHub’s web based tools.

 
% Let’s walk through an example

 
% So let’s improve the program and submit it back to the project as a proposed change.

 
% we click the Fork button as mentioned earlier to get our own copy of the project.

 
% We will clone it locally, create a topic branch, make the code change and finally push that change back up to GitHub.

 
% git diff --word-diff

 
% [-delay(1000);-]{+delay(3000);+}

 
% [-delay(1000);-]{+delay(3000);+}

 
% Clone our fork of the project locally

 
% Create a descriptive topic branch

 
% Make our change to the code

 
% Check that the change is good

 
% Commit our change to the topic branch

 
% Push our new topic branch back up to our GitHub fork

 
% GitHub noticed that we pushed a new

 
% topic branch up and presents us with a big green button

 
% check out our changes and open a Pull Request to the original project

 
% give our Pull Request a title and description.

 
% unified diff of all the changes that will be made should this branch get merged by the project owner

 
% Create pull request button

 
% it’s also often used in internal projects at the beginning of the development cycle

 
% the project owner can look at the suggested change and merge it

 
% reject it or comment on it

 
% on GitHub this happens online

 
% leave a comment by clicking on any of the lines

 
% Anyone can also leave general comments on the Pull Request.

 
% both commenting on a line of code

 
% leaving a general comment in the discussion section

 
% with GitHub you simply commit to the topic branch again and push, which will automatically update the Pull Request

 
% Adding commits to an existing Pull Request doesn’t trigger a notification

 
% This button only shows up if you have write access to the repository and a trivial merge is possible

 
% If you click it GitHub will perform a “non-fast-forward” merge, meaning that even if the merge could be a fast-forward, it will still create a merge commit.

 
% This is the basic workflow that most GitHub projects use

 
% Topic branches are created, Pull Requests are opened on them, a discussion ensues, possibly more work is done on the branch and eventually the request is either closed or merged.

 
% initiate the code review and discussion process

 
% No forking necessary

 
% creating, maintaining and administering your own project.

 
% Let’s create a new repository to share our project code with

 
% All you really have to do here is provide a project name;

 
% click the “Create Repository” button

 
% new repository on GitHub,

 
% Since you have no code there yet, GitHub will show you instructions for how to create a brand-new Git repository, or connect an existing Git project.

 
% Now that your project is hosted on GitHub, you can give the URL to anyone you want to share your project with.

 
% It is often preferable to share the HTTPS based URL for a public project

 
% The HTTPS one is also exactly the same URL they would paste into a browser to view the project there.

 
% If you’re working with other people who you want to give commit access to, you need to add them as “collaborators”

 
% you want to give them push access to your repository, you can add them to your project.

 
% “push” access

 
% both read and write access to the project and Git repository

 
% Then select “Collaborators”

 
% You can repeat this as many times as you like to grant access to everyone you like

 
% revoke access, just click the “X”

 
% get a Pull Request yourself.

 
% Pull Requests can either come from a branch in a fork of your repository or they can come from another branch in the same repository.

 
% The only difference is that the ones in a fork are often from people where you can’t push to their branch and they can’t push to yours,

 
% with internal Pull Requests generally both parties can access the branch.

 
% The other interesting URLs are the .diff and .patch URLs, which as you may guess, provide unified diff and patch versions of the Pull Request.

 
% As we covered in The GitHub Flow, you can now have a conversation with the person who opened the Pull Request.

 
% comment on specific lines of code

 
% Once the code is in a place you like and want to merge it in, you can either pull the code down and merge it locally, either with the git pull <url> <branch> syntax we saw earlier, or by adding the fork as a remote and fetching and merging.

 
% If you decide you don’t want to merge it, you can also just close the Pull Request and the person who opened it will be notified.

 
% neat trick tha

 
% If you see a Pull Request that is moving in the right direction and you have an idea for a change that depends on it or you’re not sure is a good idea, or you just don’t have push access to the target branch, you can open a Pull Request directly to it.

 
% Mentions and Notifications

 
% GitHub also has a pretty nice notifications system

 
% In any comment you can start typing a @ character and it will begin to autocomplete with the names and usernames of people who are collaborators or contributors in the project.

 
% pulling people into conversations rather than making them poll

 
% You will also be subscribed to something if you opened it, if you’re watching the repository or if you comment on something

 
% no longer wish to receive notifications, there is an “Unsubscribe” button

 
% If you go to the “Notification center” tab from the settings page, you can see some of the options you have.

 
% The two choices are to get notifications over “Email” and over “Web” and you can choose either,

 
% Web notifications only exist on GitHub and you can only check them on GitHub

 
% If you click on that, you will see a list of all the items you have been notified about

 
% Email notifications are the other way you can handle notifications through GitHub

 
% It’s also worth noting that if you have both email and web notifications enabled and you read the email version of the notification, the web version will be marked as read as well if you have images allowed in your mail client.

 
% There are a couple of special files that GitHub will notice if they are present in your repository.

 
% The first is the README file, which can be of nearly any format that GitHub recognizes as prose. For example, it could be README, README.md, README.asciidoc, etc. If GitHub sees a README file in your source, it will render it on the landing page of the project.

 
% Many teams use this file to hold all the relevant project information for someone who might be new to the repository or project. This generally includes things like:

 
% What the project is for

 
% How to configure and install it

 
% An example of how to use it or get it running

 
% The license that the project is offered under

 
% How to contribute to it

 
% Since GitHub will render this file, you can embed images or links in it for added ease of understanding.

 
% Generally there are not a lot of administrative things you can do with a single project, but there are a couple of items that might be of interest.

 
% If you would like to transfer a project to another user or an organization in GitHub, there is a “Transfer ownership” option at the bottom of the same “Options” tab of your repository settings page that allows you to do this.

 
% This is helpful if you are abandoning a project and someone wants to take it over

 
% move it into an organization.

 
% sets up a redirect from your URL

 
% In addition to single-user accounts, GitHub has what are called Organizations

 
% Organizational accounts have a namespace where all their projects exist

 
% These accounts represent a group of people with shared ownership of projects

 
% many tools to manage subgroups of those people

 
% companies

 
% New organization” from the menu

 
% First you’ll need to name your organization and provide an email address for a main point of contact for the group

 
% When you create new repositories you can create them either under your personal account or under any of the organizations that you are an owner in

 
% Organizations are associated with individual people by way of teams

 
% grouping of individual user accounts and repositories within the organization

 
% Teams make this easy, without having to manage the collaborators for every individual repository.

 
% ou can use to add members to the team

 
% Each team can have read only, read/write or administrative access to the repositorie

 
% “Settings” button

 
% Additionally, team @mentions (such as @acmecorp/frontend) work much the same as they do with individual users, except that all members of the team are then subscribed to the thread

 
% Organizations also give owners access to all the information about what went on under the organization. You can go to the Audit Log tab and see what events have happened at an organization level, who did them and where in the world they were done.

 
% curl

 
% GitHub user

 
% how to create an account

 
% manage an organization,

 
% create and push to repositories

 
% contribute to other people’s projects

 
% ccept contributions from others

 
% ource code control

 
% basic tasks of tracking and committing files

 
% important actions occur

\section{Extra}
%\input{includes/extra.tex}

\section{Conclusion}
%\input{includes/conclusion.tex}

\section*{References}
% bibliografia di riferimento
%% Ciotti
%% Burnard
%% TEI guide lines
%% Pierazzo (due libri)
%% Slide (del corso)
%% XML specification e technical report W3C (https://www.w3.org/TR/xml/)
%% XML visual
%% XSL XPATH
%% XSD (art of XSD - SQL validation)
%% DTD (libro visual XML)
%% RELAXNG (libro relaxng, tutorial)

%bibliografia
\begin{frame}
    \frametitle{References}
    \addtocounter{nframe}{1}
    \begin{thebibliography}{10}
        \setbeamertemplate{bibliography item}[paper]
        \tiny\bibitem{GITPRO2014} Chacon, S., e B. Straub. 2014. Pro Git. Apress. 
    \end{thebibliography}

\end{frame}

\begin{frame}
    \frametitle{References}
    \addtocounter{nframe}{1}
    \begin{thebibliography}{10}
        
        \setbeamertemplate{bibliography item}[online]
        \tiny\bibitem{GITPRO} \textit{Pro Git book, written by Scott Chacon and Ben Straub}, 2nd Edition (2014). \url{https://git-scm.com/book/it/v2}
    \end{thebibliography}

\end{frame}


\end{document}